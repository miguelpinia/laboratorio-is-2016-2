% Created 2016-02-17 mié 16:55
\documentclass[11pt]{article}
\usepackage[utf8]{inputenc}
\usepackage[T1]{fontenc}
\usepackage{fixltx2e}
\usepackage{graphicx}
\usepackage{grffile}
\usepackage{longtable}
\usepackage{wrapfig}
\usepackage{rotating}
\usepackage[normalem]{ulem}
\usepackage{amsmath}
\usepackage{textcomp}
\usepackage{amssymb}
\usepackage{capt-of}
\usepackage{hyperref}
\author{Miguel Piña}
\date{\textit{[2016-02-16 mar 10:22]}}
\title{Introducción a HTML y CSS}
\hypersetup{
 pdfauthor={Miguel Piña},
 pdftitle={Introducción a HTML y CSS},
 pdfkeywords={},
 pdfsubject={},
 pdfcreator={Emacs 24.5.4 (Org mode 8.3.3)},
 pdflang={English}}
\begin{document}

\maketitle
\setcounter{tocdepth}{1}
\tableofcontents





\section*{¿Qué es HTML?}
\label{sec:orgheadline1}

\begin{itemize}
\item HTML siglas de HyperText Markup Language («lenguaje de marcas de hipertexto»).
\item Sirve para la elaboración de páginas web.
\item Es un estándar para la escritura de páginas web.
\item Está a cargo de la W3C.
\item El procesamiento de las páginas web recae en los navegadores.
\item La estandarización permite que sea interpretada por cualquier navegador web.
\end{itemize}

\section*{Estructura de un documento HTML}
\label{sec:orgheadline2}

\begin{itemize}
\item Ejemplo
\end{itemize}

\begin{verbatim}
<!DOCTYPE HTML>
<html>
  <head>
    <title>Ejemplo1</title>
  </head>
  <body>
    <p>Párrafo de ejemplo</p>
  </body>
</html>
\end{verbatim}

\section*{Etiquetas básicas}
\label{sec:orgheadline3}

\begin{itemize}
\item <html>: define el inicio del documento HTML, le indica al navegador que lo que
viene a continuación debe ser interpretado como código HTML. Esto es así de
facto, ya que en teoría lo que define el tipo de documento es el DOCTYPE, que
significa la palabra justo tras DOCTYPE el tag de raíz.

\item <script>: incrusta un script en una web, o llama a uno mediante src="url del
script". Se recomienda incluir el tipo MIME en el atributo type, en el caso de
JavaScript text/javascript.

\item <head>: define la cabecera del documento HTML; esta cabecera suele contener
información sobre el documento que no se muestra directamente al usuario como,
por ejemplo, el título de la ventana del navegador.
\end{itemize}

\section*{Etiquetas básicas}
\label{sec:orgheadline4}

\begin{itemize}
\item <title>: define el título de la página. Por lo general, el título aparece en la
barra de título encima de la ventana.

\item <link>: para vincular el sitio a hojas de estilo o iconos. Por ejemplo:
<link rel="stylesheet" href="/style.css" type="text/css">.

\item <style>: para colocar el estilo interno de la página; ya sea usando CSS u otros
lenguajes similares. No es necesario colocarlo si se va a vincular a un archivo
externo usando la etiqueta <link>.

\item <meta>: para metadatos como la autoría o la licencia, incluso para indicar
parámetros http (mediante http-equiv="") cuando no se pueden modificar por no
estar disponible la configuración o por dificultades con server-side scripting.
\end{itemize}

\section*{Etiquetas básicas}
\label{sec:orgheadline5}

\begin{itemize}
\item <body>: define el contenido principal o cuerpo del documento. Esta es la parte
del documento html que se muestra en el navegador; dentro de esta etiqueta
pueden definirse propiedades comunes a toda la página, como color de fondo y
márgenes. Dentro del cuerpo <body> es posible encontrar numerosas etiquetas. A
continuación se indican algunas a modo de ejemplo:

\item <h1> a <h6>: encabezados o títulos del documento con diferente relevancia.

\item <table>: define una tabla.

\item <tr>: fila de una tabla.

\item <td>: celda de una tabla (debe estar dentro de una fila).

\item <a>: hipervínculo o enlace, dentro o fuera del sitio web. Debe definirse el
parámetro de pasada por medio del atributo href. Por ejemplo: <a
href="\url{http://www.example.com}" title="Ejemplo" target="\(_{\text{blank}}\)"
tabindex="1">Ejemplo</a> se representa como Ejemplo).
\end{itemize}

\section*{Etiquetas básicas}
\label{sec:orgheadline6}

\begin{itemize}
\item <div>: división de  la página. Se recomienda,  junto con css, en  vez de <table>
cuando se desea alinear contenido.

\item <img>: imagen. Requiere del atributo src, que indica la ruta en la que se
encuentra la imagen. Por ejemplo: <img src="./imágenes/mifoto.jpg" />. Es
conveniente, por accesibilidad, poner un atributo alt="texto alternativo".

\item <li><ol><ul>: etiquetas para listas.

\item <b>: texto en negrita (etiqueta desaprobada. Se recomienda usar la etiqueta
\end{itemize}
<strong>).

\begin{itemize}
\item <i>: texto en cursiva (etiqueta desaprobada. Se recomienda usar la etiqueta <em>).
\end{itemize}

\section*{Etiquetas básicas}
\label{sec:orgheadline7}

\begin{itemize}
\item <s>: texto tachado (etiqueta desaprobada. Se recomienda usar la etiqueta <del>).

\item <u>: Antes texto subrayado. A partir de HTML 5 define porciones de texto
diferenciadas o destacadas del resto, para indicar correcciones por
ejemplo. (etiqueta desaprobada en HTML 4.01 y redefinida en HTML 5)
\end{itemize}
\end{document}